\section{Elecció de paràmetres $S/C$ i $\Psi$}
Es parteix del coneixement que el treball específic que ha de subministrar el nostre motor és $\tau_{23} = 300000 J/kg$.\\

A partir de motors similars, s'aprecia que les solucions per compressors estan entre 7, 8, 9 i 10 graons. Per tant, per cada un dels 4 casos es pot trobar el valor que es tindria de solidesa i el coeficient de flux del gràfic de $\tau_{esc}$. S'escollirà el cas que tingui un major rendiment per escaló de tots els possibles. \\

Primer de tot es necessari saber el treball específic que ha de realitzar cada etapa de compressió segons el numero total d'etapes. Es calcula com $\tau_{esc}=\tau_{23}/N$ on $N$ és el número d'etapes de compressió.

\begin{longtable}[H]{C{1.5cm} C{2.5cm}}
	\toprule[2pt]
	\textbf{N} &  \textbf{$\tau_{esc} \quad [J/kg]$} \\ \bottomrule[2pt]
	7 & $4.29\mathrm{x}10^4$\\ \midrule
	8 & $3.75\mathrm{x}10^4$\\ \midrule
	9 & $3.33\mathrm{x}10^4$\\ \midrule
	10&$3.00\mathrm{x}10^4$
	\\ \bottomrule[2pt]
	\caption{Treball específic segons etapes de compressió}
	\label{valorsI}
\end{longtable}

Després, es superposen (Figura \ref{TAUS}) els resultats obtinguts per cada escaló segons nombre d'etapes amb els valors de $\tau$ inicialment calculats per diferents $S/C$ i $\Psi$.\\
\begin{figure}[H]
	\centering
	\includegraphics[width=\textwidth]{./code/figures/parametres/TAUSesg}
	\caption{Valors de $\tau$ en funció de $S/C$ i $\Psi$.}
	\label{TAUS}
\end{figure}

Ara, es busca l'intersecció dels treballs específics calculats per a cada etapa amb els treballs d'escaló calculats per per diferents valors de $S/C$ i $\Psi$.

El resultat, permet veure quin paràmetre $S/C$ es el millor per a cada punt d'intersecció, però per saber el flux ($\Psi$) caldrà interpolar entre els dos punts més propers a l'intersecció.\\

S'han interpolat els valors del flux ($\Psi$) linealment a partir de dos punts d'informació propers a l'intersecció, $(x_a,y_a)$ i $(x_b,y_b)$, per obtenir un tercer punt interpolat $(x,y)$ segons,
\begin{equation}
	y = y_a + (x-x_a)\frac{(y_b-y_a)}{(x_b-x_a)}
\end{equation}
per aquest cas particular,
\begin{equation}
\Psi_N = \Psi_a + (\tau_{N}-\tau_a)\frac{(\Psi_b-\Psi_a)}{(\tau_b-\tau_a)}
\end{equation}

Aquesta aproximació lineal, és vàlida ja que es treballa en un interval petit entre les dues dades conegudes.

\begin{longtable}[H]{C{1.5cm} C{1.5cm} C{1.5cm}}
	\toprule[2pt]
	\textbf{N} &  \textbf{$S/C$}  & \textbf{$\Psi$}\\ \bottomrule[2pt]
	7 & -- & --\\ \midrule
	8 & 0.6 &0.5437\\ \midrule
	9 & 0.8 &0.6121\\ \midrule
	10 & 1 &0.6455
	\\ \bottomrule[2pt]
	\caption{Paràmetres escollits}
\end{longtable}

