\section{Introducció, objectius i hipòtesis}
\label{Intro}
El present treball forma part de l'assignatura de Sistemes de Propulsió d'Aeronaus. En la segona part de l'assignatura es profunditza en el temari relacionat amb turbomaquinaria. Per tal d'assolir els coneixements adequats, es proposa realitzar un treball amb l'objectiu de dissenyar un compressor.\\
Les condicions i requisits pel disseny del compressor son els següents:
\begin{itemize}
\item Treball total del compressor: $\tau_{23}=300 kJ/kg$
\item Consum del compressor: $G=28 kg/s$
\item Pressió atmosfèrica: $P_{at}=1kg/cm^2$
\item Temperatura atmosfèrica: $T_{at}=288K$
\item Constant dels gasos ideals: $286.8 J/kgK$
\item Exponent: $\gamma =1.4 $
\end{itemize}
S'han de suposar certes hipòtesis per a poder dur a terme el disseny del compressor. Aquestes hipòtesis son: 
\begin{itemize}
\item Màquina periòdica. El treball per esglaó es constant
\item Grau de reacció $R=0.5$
\item Angle de lliscament $\delta = 0$
\item Radi mitjà constant
\item Velocitat axial constant 
\end{itemize}
Amb això podem donar pas al càlcul del compressor. 